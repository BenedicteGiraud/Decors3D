\section{Conclusion}
%%%%%%%%%%%%%%%
 % Frame Section title
 \begin{frame}
 \title{Conclusion}
 \titlepage

 \end{frame}
 

%%%%%%%%%%%%%%%%%%%
 % Frame conclusion
\begin{frame}
\frametitle{Conclusion}
\emph{Qu'est-ce qu'un décor ?}
\begin{enumerate}[-]
\item Une partie d'un film fixe ou en mouvement uniforme
\end{enumerate}
\emph{Comment trouver les données utiles pour les trouver ?}
\begin{enumerate}[-]
\item Séquencer le film, trouver les images sans visage et sans flou
\end{enumerate}
\emph{Comment supprimer le non-décor ?}
\begin{enumerate}[-]
\item Repérer les élements qui bougent, faire de l'inpainting 2D+t
\end{enumerate}
\emph{Comment reconstruire le décor en 3D ?}
\begin{enumerate}[-]
\item Faire correspondre des points clés, retrouver la position des caméras
\end{enumerate}
\end{frame}

\begin{frame}
\frametitle{Travaux restant}
\begin{enumerate}
\item Retrouver les lieux à des moments différents du film
\item Meilleure estimation du mouvement principal
\item Inpainting 3D
\item Meilleur agencement des parties
\end{enumerate}
\end{frame}



%%%%%%%%%%%%%%%%%%%
 % Frame biblio
\nocite{*}
\begin{frame}[allowframebreaks]
    \bibliographystyle{apalike}
    \bibliography{references.bib}
\end{frame}


%%%%%%%%%%%%%%%%%%%
 % Frame question
 \begin{frame}
 \frametitle{Questions}
 \begin{figure}
 \includegraphics[width=0.5\textwidth]{Fig/question.jpg}
 \end{figure}

\end{frame}

