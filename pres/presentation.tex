% chinthakawk@gmail.com

\documentclass[compress]{beamer} %this is to use copressed header from the
\usepackage{etex}
\usepackage{beamerthemeshadow} %our theme, later you can change it
\usepackage[frenchb]{babel}
\usepackage{fancyhdr} % Required for custom headers
\usepackage[utf8]{inputenc}
\usepackage[lined,boxed]{algorithm2e}
\usepackage[all]{xy}
\usepackage{animate} %need the animate.sty file
\graphicspath{{./Figure/}} 
% editing header
\setbeamertemplate{headline}
{
  \leavevmode%
  \hbox{%
  \begin{beamercolorbox}[wd=.5\paperwidth,ht=2.25ex,dp=1.8ex,leftskip=1em,left]{section in head/foot}%
    \usebeamerfont{subsection in head/foot}\hspace*{5ex}\insertsectionhead
  \end{beamercolorbox}%
  \begin{beamercolorbox}[wd=.5\paperwidth,ht=2.25ex,dp=1.8ex,left,leftskip=1em]{subsection in head/foot}%
    \usebeamerfont{section in head/foot}\insertshorttitle\hspace*{2ex}
  \end{beamercolorbox}}%
  \vskip0pt%
}

% editing footer
\setbeamertemplate{footline}
{
  \leavevmode%
  \hbox{%
  \begin{beamercolorbox}[wd=.5\paperwidth,ht=2.25ex,dp=1ex,right]{author in head/foot}%
    \usebeamerfont{author in head/foot}\insertshortauthor\hspace*{2ex}
  \end{beamercolorbox}%
  \begin{beamercolorbox}[wd=.5\paperwidth,ht=2.25ex,dp=1ex,right]{date in head/foot}%
    \usebeamerfont{date in head/foot}\insertshortdate{}\hspace*{2em}
    \insertframenumber{} / \inserttotalframenumber\hspace*{2ex} 
  \end{beamercolorbox}}%
  \vskip0pt%
}
\definecolor{orange}{rgb}{1,0.5,0}
\definecolor{darkgreen}{rgb}{0,0.5,0}
\definecolor{darkblue}{rgb}{0,0,0.5}

\def\etal{et al.\ }
\def\ie{i.e.\ }
\def\eg{e.g.\ }

%insert 2 figures on a row
\newcommand{\insertTwoF}[4]{
  \begin{figure}[h!]
    \centering
    \begin{minipage}{#4\linewidth}
    \includegraphics[width=\linewidth]{#1}
    \end{minipage}
    \begin{minipage}{#4\linewidth}
    \includegraphics[width=\linewidth]{#2}
    \end{minipage}
      \caption{#3}
  \end{figure}  
}

\newcommand{\insertF}[3]{
  \begin{figure}[h!]
    \centering
    \begin{minipage}{#3\linewidth}
    \includegraphics[width=\linewidth]{#1}
    \end{minipage}  
      \caption{#2}
  \end{figure}  
}

%enable numbering captions for the images
\setbeamertemplate{caption}[numbered]


\begin{document}

 \title{Décors3D} 
 \section{title}
 %title page
 \begin{frame}
\titlepage
    \centering
    \begin{minipage}{0.4\textwidth}
    \begin{flushleft} \large
    \emph{Authors:}\\
    T. Dalens, A. Dhobb,\\
    B. Giroud, M. Keriben,\\
    T. Rebele, M. Xu\\
    \end{flushleft}
    \end{minipage}
    \begin{minipage}{0.4\textwidth}
    \begin{flushright} \large
    \emph{Course:}\\
    SI 381
    \end{flushright}
    \end{minipage}\\[3cm]
 \end{frame}
 \section{Segmentation en plans}
 %MODIFIER ICI
 
 \begin{frame}
 \frametitle{Architecture}
 \textbf{Entrée :}  Une vidéo \\ 
 
 \textbf{Sortie :}  Une série de frames clés pour chaque plan de la vidéo \\
 
 \end{frame}

 \begin{frame}
 \frametitle{Histogramme HSV}
 \insertTwoF{Fig/hsv1.jpg}{Fig/hsv2.jpg}{décomposition selon H,S et V}{0.3}
 \end{frame}
 
 \begin{frame}
 \frametitle{Histogramme HSV}
 On calcule un histogramme HSV normalisé sur 16 bins pour chaque frame : 8 bins pour H, 4 pour S et 4 pour V.\\
 \vspace{1cm}
 \verb![Rasheed, Shah, CVPR 2003]!\\
 \end{frame}
 
 \begin{frame}
 \frametitle{Détection des changements plans}
 
 \textbf{Principe :} On compare les histogrammes entre 2 frames consécutives. Si l'intersection entre ces 2 histogrammes est inférieure à un certain seuil, cela indique qu'il y a un changement de plan. \\
 \[d_{intersection}(H_{1},H_{2}) = \sum_{i} min(H_{1}(i),H_{2}(i))\]
 
 \end{frame}
 
 
 \begin{frame}
 \frametitle{Extraction de frames clés d'une même scène}
 \textbf{Principe :}
\begin{itemize} 
\item{On fixe un deuxième seuil}
\item{On extrait la première frame du plan qui est la frame de référence}
\item{Pour chaque frame de ce plan, on calcule l'intersection de son histogramme avec l'histogramme de la frame de référence}
\item{Si l'intersection est inférieure à notre seuil, on extrait cette frame qui devient la nouvelle frame de référence}
\end{itemize}
\end{frame}

\begin{frame}
 \frametitle{Extraction de frames clés d'une même scène}
\end{frame}

 

 
 
 
 \section{Inpainting}
 %MODIFIER ICI
 \section{Modélisation 3D}
 %MODIFIER ICI
 \section{Examples}
 
 \begin{frame}
  \frametitle{Exemples d'images}
  \insertF{Fig/chat1.jpg}{petit chaton 1}{0.3}
  \insertF{Fig/chat2.jpg}{petit chaton 2}{0.3}
 \end{frame}
 
 \begin{frame}
  \frametitle{Exemples d'images}
  \insertTwoF{Fig/chat1.jpg}{Fig/chat2.jpg}{petits chatons}{0.4}
 \end{frame}
 \begin{frame}
   Exemples de table des matières

 \begin{enumerate}


   \item partie 1
   \item partie 2
   \item partie 3
   
 \end{enumerate}

 
 \end{frame} 
 
 \begin{frame}
 exemples de texte avec \emph{emphase}, \textbf{gras}, maths $x_{n+1}^2=0$ et encore maths
 \[
 \frac{\sin(x)}{x} = 6
 \]
 \end{frame}
 
\end{document}