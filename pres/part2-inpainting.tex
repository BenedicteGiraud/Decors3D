 \section{Inpainting}
 %MODIFIER ICI
 %%%%%%%%%%%%%%%
 % Frame Section title
 \begin{frame}
 \title{Inpainting}
 \titlepage

    \begin{minipage}{0.3\textwidth}
    \begin{flushleft} \large
    \emph{Binôme :}\\
    M. Keribin\\
    T. Rebele
    \end{flushleft}
    \end{minipage}
    \begin{minipage}{0.5\textwidth}
    \begin{flushright} \large
    \begin{figure}
    \includegraphics[width=1.4\textwidth]{Fig/architectureSectionInpainting.png}
    \end{figure}
    \end{flushright}
    \end{minipage}\\[3cm]
    
 \end{frame}



 %%%%%%%%%%%%%%%
 % Frame Architecture inpainting
\begin{frame}
  \frametitle{Architecture du programme}
  \insertF{Fig/architectureInpainting.png}{Architecture de la partie Inpainting}{1}

\end{frame}



 %%%%%%%%%%%%%%%
 % Frame KeyPoints
\begin{frame}
  \frametitle{Keypoints}
  
  \begin{itemize}
  \item SURF
  	\begin{itemize}
  	\item
  	\end{itemize}
  	
  \item GFTT (Harris)
	\begin{itemize}
  	\item
  	\end{itemize}
  	
  \item Canny
    \begin{itemize}
  	\item
  	\end{itemize}
  	
  \end{itemize}
    \insertF{Fig/cannyPoints.png}{Points de Canny}{0.4}


\end{frame}



 %%%%%%%%%%%%%%%
 % Frame create trace
\begin{frame}
  \frametitle{Trace}
  \begin{itemize}
  \item Recherche de correspondances entre points
  	\begin{itemize}
  	\item Sélection des keypoints
  	\item Recherche dans le voisinage frame suivante
  	\end{itemize}
  \item Estimer l'homographie
  \item Trace
  	\begin{itemize}
  	\item Créer et continuer la trace
  	\item Classifier la trace
  	\begin{itemize}
  		\item fixe
  		\item mobile
  		\item indéterminé
  	\end{itemize}
  	\end{itemize}
  \end{itemize}
  
    \insertF{Fig/cannyKeypoints.png}{Mise en évidence des keypoints et de la trace}{0.4}


\end{frame}


 %%%%%%%%%%%%%%%
 % Frame inpainting principe
\begin{frame}
  \frametitle{Remplissage}
  
  \begin{itemize}
  \item Classifier les zones
  	\begin{itemize}
  	\item fixe $\leftrightarrow$ noir
  	\item mobile $\leftrightarrow$ couleur du fond
  	\end{itemize}
  \item Remplir quand le point devient fixe  
  \end{itemize}

  \begin{figure}
  \includegraphics[width=0.45\textwidth]{Fig/bunny2-masked.png}
  \includegraphics[width=0.45\textwidth]{Fig/bunny1-original.png}
  \caption{Principe de l'inpainting 2D + t}
  \end{figure}
   
\end{frame}


 %%%%%%%%%%%%%%%
  % Frame demonstration
 \begin{frame}
   \frametitle{Démonstration}
   \begin{figure}
   \includegraphics[width=0.5\textwidth]{Fig/workInProgress.png}
   \end{figure}

 \end{frame}