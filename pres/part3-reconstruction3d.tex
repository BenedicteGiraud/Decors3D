
 \section{Modélisation 3D}
\subsection{Entrée}
	  
	  \begin{frame}
	  \frametitle{Entrée du programme}
	  On prend en entrée:
	  \begin{itemize}
	    
	    \setbeamertemplate{itemize item}[triangle]
	    \item Des images d’une même scène décalée dans l’espace
	    \end{itemize}

	    
	  \end{frame}
	  
	  \subsection{Traitement}
	  %\subsection{Bundler}
	  
	  \begin{frame}
	  \frametitle{Bundler}
	   Bundler:
	  \begin{itemize}
	    \setbeamertemplate{itemize item}[triangle]
	    \item Calcule les SIFT pour trouver des correspondances
	    
	    \setbeamertemplate{itemize item}[triangle]
	    \item Estime la position des caméras
	    
	    \setbeamertemplate{itemize item}[triangle]
	    \item Calcule un nuage de points 3D dense
	    \end{itemize}
	  \end{frame}
	  
	  %\subsection{MeshLab}
	  
	  \begin{frame}
	  \frametitle{MeshLab}
	    MeshLab:
	  \begin{itemize}
	    \setbeamertemplate{itemize item}[triangle]
	    \item Récupère le nuage de points calculé
	    
	    \setbeamertemplate{itemize item}[triangle]
	    \item Construit un maillage
	    
	    \setbeamertemplate{itemize item}[triangle]
	    \item Calcule une couleur par sommet du maillage
	    \end{itemize}
	 \end{frame}

	 \subsection{Sortie}
	 \begin{frame}
	  \frametitle{Sortie du programme}
	   Nous avons donc en sortie:
	  \begin{itemize}
	    \setbeamertemplate{itemize item}[triangle]
	    \item Un maillage avec les couleurs par scène données en entrée.
	    
	    
	    \end{itemize}
	 \end{frame}
