 \section{Segmentation}
 %MODIFIER ICI
 
 \begin{frame}
 \frametitle{Architecture}
 \textbf{Entrée :}  Une vidéo \\ 
 
 \textbf{Sortie :}  Une série de frames clés pour chaque plan de la vidéo \\
 
 \end{frame}

 \begin{frame}
 \frametitle{Histogramme HSV}
 %\insertTwoF{Fig/hsv1.jpg}{Fig/hsv2.jpg}{décomposition selon H,S et V}{0.3}
 \end{frame}
 
 \begin{frame}
 \frametitle{Histogramme HSV}
 On calcule un histogramme HSV normalisé sur 16 bins pour chaque frame : 8 bins pour H, 4 pour S et 4 pour V.\\
 \vspace{1cm}
 \verb![Rasheed, Shah, CVPR 2003]!\\
 \end{frame}
 
 \begin{frame}
 \frametitle{Détection des changements plans}
 
 \textbf{Principe :} On compare les histogrammes entre 2 frames consécutives. Si l'intersection entre ces 2 histogrammes est inférieure à un certain seuil, cela indique qu'il y a un changement de plan. \\
 \[d_{intersection}(H_{1},H_{2}) = \sum_{i} min(H_{1}(i),H_{2}(i))\]
 
 \end{frame}
 
 
 \begin{frame}
 \frametitle{Extraction de frames clés d'une même scène}
 \textbf{Principe :}
\begin{itemize} 
\item{On fixe un deuxième seuil}
\item{On extrait la première frame du plan qui est la frame de référence}
\item{Pour chaque frame de ce plan, on calcule l'intersection de son histogramme avec l'histogramme de la frame de référence}
\item{Si l'intersection est inférieure à notre seuil, on extrait cette frame qui devient la nouvelle frame de référence}
\end{itemize}
\end{frame}

\begin{frame}
 \frametitle{Extraction de frames clés d'une même scène}
\end{frame}