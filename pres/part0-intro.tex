\section{Introduction}
 %%%%%%%%%%%%%%%%%%%
 % Frame presentation of the programm : idee, what he does
 \begin{frame}
   \frametitle{Introduction - Contexte}
   
   \begin{itemize}
   
   \item Industrie du film est très importante et donne de nombreux produits dérivés
   	\begin{itemize}
   	\item Parodies
   	\item Industrie des jeux vidéos
   	\item Reconstruction décor réel en virtuel
   	\end{itemize}
   
   \item État de l'art
   	\begin{itemize}
   	\item Indexation
   	\item Détection de personnages
   	\item Inpainting 2D+t
   	\item Reconstruction 3D d'objets

   	\end{itemize}
   	
   \item Notre idée
   	\begin{itemize}
   	\item Supprimer les personnages pour obtenir les décors des films
   	\end{itemize}
   
   \end{itemize}
 
 \end{frame}


 %%%%%%%%%%%%%%%%%%%
 % Frame plan
\begin{frame}
  \frametitle{Introduction - Plan de la présentation}
  \begin{itemize}
  \item Introduction
  \end{itemize}
  \begin{enumerate}
  \item Segmentation
  \item Inpainting 2D+t
  \item Reconstruction 3D
  \end{enumerate}
  \begin{itemize}
  \item Conclusion
  \item Démonstration
  \end{itemize}
  

\end{frame}


 %%%%%%%%%%%%%%%
  % Frame purposes, how to use it
 \begin{frame}
   \frametitle{Introduction - Objectif}
   
   \begin{itemize}
   \item Obtenir le décor des films
   	\begin{itemize}
   	\item Sous format 3D si possible
   	\item En 2D sinon
   	\end{itemize}
   	
   \item Fonctionnement idéal
   	\begin{itemize}
   	\item Donner un film en entrée
   	\item Détecter les différents points de vue
   	\item Grouper les scènes d'un même décor
   	\item Supprimer les personnages
   	\item Obtenir un maillage 3D du décor
   	\end{itemize}
   
   \item Problèmes à résoudre
  	\begin{itemize}
	\item Présence de personnages devant le décor
	\item Définition du décor
	\item Scène fixes
  	\end{itemize}

   \end{itemize}
	
 \end{frame}
 
 %%%%%%%%%%%%%%%%%%%
 % Frame architecture
\begin{frame}
  \frametitle{Introduction - Architecture}
  \insertF{Fig/architectureGlobale.png}{Architecture de notre programme}{1}

\end{frame}


 %%%%%%%%%%%%%%%
  % Frame demonstration
 \begin{frame}
   \frametitle{Introduction - Démonstration}
   \insertF{Fig/workInProgress.png}{Lancement de la démonstration en parallèle}{0.5}


 \end{frame}